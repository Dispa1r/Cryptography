\documentclass[12pt,letterpaper,final]{report}
\usepackage[utf8]{inputenc}
\usepackage{amsmath}
\usepackage{amsfonts}
\usepackage{amssymb}
\usepackage{amsthm}
\usepackage{enumitem}
\usepackage{seqsplit}
\usepackage{mathtools}
\usepackage{hyperref}
\renewcommand\qedsymbol{$\blacksquare$}
\usepackage{enumerate}


\begin{document}
\fbox{
\vbox{
\begin{flushleft}
Tanner Krebs, Gerardo Lopez, William Thornton \\  % authors' names
Math 314 \\  %class
Due: 10/26/17 \\  % date
\end{flushleft}
\center{\Large{\textbf{COSC/MATH - 314}}}
\center{\Large{\textbf{Assignment 6}}}
}  % end vbox
} % end fbox

\vline


\noindent\textbf{Problem 1}: . Find $-a$ (the additive inverse) for each element a in $Z_{5}$.  Find $a^{-1}$   (the multiplicative inverse) of each element a, except 0, in $Z_{5}$. 
(Recall that $Z_{5}$   is the class of residues mod 5, together with the operations of $+ (mod 5) and * (mod 5))$.
 

\bigskip
\noindent\textbf{Problem 2}: Using the extended Euclidean algorithm find the multiplicative inverses of

\newlist{alphalist}{enumerate}{1}
\setlist[alphalist,1]{label=\textbf{(\alph*.)}}
\begin{alphalist}
	\item 1234 mod 4321. Show calculations
	\item 550 mod 1769. Show calculations
	
\end{alphalist}

\bigskip
Verify that gcd(1234, 4321) = 1.

4321 = 1234(3) + 619	619 = 4321 - 1234(3)	
1234 = 619(1) + 615	615 = 1234 - 619(1)	
619 = 615(1) + 4		4 = 619 - 615(1)	
615 = 4(153) + 3		3 = 615 - 4(153)	
4 = 3(1) + 1			1 = 4 - 3(1)
3 = 1(3) + 0  

----------------------------

1 = 4 - 3(1)
1 = 4 - (615 - 4(153)) 
1 =  154 (4) - 615
 = 154 (619 - 615) - 615
1 = 154 (619) - 155 (615)
1 = 154 (619) - 155 (1234-619)
1 = 309 (619) - 155 (1234)
1 = 309 (4321-3 (1234)) - 155 (1234)
1 = 309 (4321)-1082 (1234)

----------------------------

x = - 1082
x = (-1082) + 4321 = 3239

1234 mod 4321 = 3239

\bigskip

Verify that gcd(550, 1769) = 1.

1769 = 550(3) + 119	119 = 1769 - 550(3)
550 = 119(4) + 74		74 = 550 - 119(4)
119 = 74(1) + 45		45 = 119 - 74(1)
74 = 45(1) + 29		29 = 74 - 45(1)
29 = 16(1) + 13		13 = 29 - 16(1)
16 = 13(1) + 3		3 = 16 - 13(1)
13 = 3(4) + 1		1 = 13 - 4(1)
3 = 1(3) + 0

----------------------------


----------------------------

x = 

550 mod 1769 = 550

\bigskip
\noindent\textbf{Problem 3}: Exercise 1 (a) and (b), page 104, textbook.


\bigskip
\noindent\textbf{Problem 4}: Exercise 3 (a) and (b), page 104, textbook (use the method on page 74, which we also discussed in class).


\bigskip
\noindent\textbf{Problem 5}: Exercise 5(a) and 5(b), page 104.


\end{document}
