\documentclass[12pt,letterpaper,final]{report}
\usepackage[utf8]{inputenc}
\usepackage{amsmath}
\usepackage{amsfonts}
\usepackage{amssymb}
\usepackage{amsthm}
\usepackage{enumitem}
\usepackage{seqsplit}
\usepackage{mathtools}
\usepackage{hyperref}
\renewcommand\qedsymbol{$\blacksquare$}
\usepackage{enumerate}


\begin{document}
\fbox{
\vbox{
\begin{flushleft}
Tanner Krebs, Gerardo Lopez, William Thornton \\  % authors' names
Math 314 \\  %class
Due: 10/26/17 \\  % date
\end{flushleft}
\center{\Large{\textbf{COSC/MATH - 314}}}
\center{\Large{\textbf{Assignment 6}}}
}  % end vbox
} % end fbox

\vline


\noindent\textbf{Problem 1}: . Find $-a$ (the additive inverse) for each element a in $Z_{5}$.  Find $a^{-1}$   (the multiplicative inverse) of each element a, except 0, in $Z_{5}$. 
(Recall that $Z_{5}$   is the class of residues mod 5, together with the operations of $+ (mod 5) and * (mod 5))$.
 

\bigskip
\noindent\textbf{Problem 2}: Using the extended Euclidean algorithm find the multiplicative inverses of

\newlist{alphalist}{enumerate}{1}
\setlist[alphalist,1]{label=\textbf{(\alph*.)}}
\begin{alphalist}
	\item 1234 mod 4321. Show calculations
	\item 550 mod 1769. Show calculations
	
\end{alphalist}

\bigskip
$1234$ (mod 4321)  \newline

To start, we need to Verify that gcd(1234, 4321) = 1.\newline

4321 = 1234(3) + 619\newline	
1234 = 619(1) + 615\newline			
619 = 615(1) + 4\newline				
615 = 4(153) + 3\newline		
4 = 3(1) + 1	\newline		
3 = 1(3) + 0  \newline

Now we shift the equations above to get our equations below: \newline

619 = 4321 - 1234(3) \newline 
615 = 1234 - 619(1) \newline
4 = 619 - 615(1) \newline
3 = 615 - 4(153)	 \newline
1 = 4 - 3(1) \newline

If we use substitution we get: \newline

1 = 4 - 3(1)  \newline
1 = 4 - (615 - 4(153))  \newline
1 =  154 (4) - 615 \newline
1 = 154 (619 - 615) - 615 \newline
1 = 154 (619) - 155 (615) \newline
1 = 154 (619) - 155 (1234-619) \newline
1 = 309 (619) - 155 (1234) \newline
1 = 309 (4321-3 (1234)) - 155 (1234) \newline
1 = 309 (4321)-1082 (1234) \newline

Finally we can note that: \newline

x = - 1082 \newline
x = (-1082) + 4321 = 3239 \newline


Hence  $1234$ (mod 4321) $\equiv$ 3239 \newline

\bigskip

$550$ (mod 1769)  \newline

To start, we need to Verify that gcd(550, 1769) = 1. \newline

1769 = 550(3) + 119 \newline	
550 = 119(4) + 74	 \newline	
119 = 74(1) + 45	 \newline	
74 = 45(1) + 29	 \newline	
29 = 16(1) + 13	 \newline	
16 = 13(1) + 3	 \newline	
13 = 3(4) + 1	 \newline	
3 = 1(3) + 0 \newline

Now we shift the equations above to get our equations below: \newline

119 = 1769 - 550(3) \newline
74 = 550 - 119(4) \newline
45 = 119 - 74(1) \newline
29 = 74 - 45(1) \newline
13 = 29 - 16(1) \newline
3 = 16 - 13(1) \newline
1 = 13 - 4(3) \newline

If we use substitution we get: \newline

1 = 13 - 4(3) \newline
1 = 13 - 4(16 - 13(1)) \newline
1 = 5(13) - 64 \newline
1 = 5(29 - 16(1)) - (4(16)) \newline
1 = 5((74 - 45(1)) - 16) - (4(16)) \newline
1 = 5((74 - 45(1)) - 16) - (4(16)) \newline


Hence  $550$ (mod 1769) $\equiv$ 550 \newline

\bigskip
\noindent\textbf{Problem 3}: Exercise 1 (a) and (b), page 104, textbook.
\bigskip

$12x \equiv 28$ (mod 236) \newline

To start, is there anything we can divide the constants? Yes, we can divide all of them by a factor of 4. 
\bigskip\newline Now we have the equation: \indent $3x \equiv 7$ (mod 59) 
\bigskip\newline 
In this case... 3 (20) = 1 (mod 59)
We can rearrange it such as...

\indent $x \equiv 20 (7) \equiv 140 \equiv 22$ (mod 59)
\bigskip\newline 
In short, $x \equiv 22$ (mod 59) which would give us our solution set. \newline
In this case, x is 2, 81, 140, and 199.
\bigskip

$12x \equiv 30$ (mod 236) \newline

To start, is there anything we can divide the constants? NO, we can't divide all of them by a factor of 4. \bigskip\newline
Since 30  doesn't divide nicely by 4, this implies that for x, $12x (mod 236)$ is divisible by 4, which is FALSE. 
 \bigskip\newline
As a result, $12x \equiv 30$ (mod 236) has NO SOLUTION.



\bigskip
\noindent\textbf{Problem 4}: Exercise 3 (a) and (b), page 104, textbook (use the method on page 74, which we also discussed in class).


\bigskip
\noindent\textbf{Problem 5}: Exercise 5(a) and 5(b), page 104.


\end{document}
