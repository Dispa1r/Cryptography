\documentclass[12pt,letterpaper,final]{report}
\usepackage[utf8]{inputenc}
\usepackage{amsmath}
\usepackage{amsfonts}
\usepackage{amssymb}
\usepackage{amsthm}
\usepackage{enumitem}
\usepackage{seqsplit}
\usepackage{mathtools}
\usepackage{hyperref}
\renewcommand\qedsymbol{$\blacksquare$}
\usepackage{enumerate}


\begin{document}
\fbox{
\vbox{
\begin{flushleft}
Tanner Krebs, Gerardo Lopez, William Thornton \\  % authors' names
Math 314 \\  %class
9/28/17 \\  % date
\end{flushleft}
\center{\Large{\textbf{COSC/MATH - 314}}}
\center{\Large{\textbf{Assignment 4}}}
}  % end vbox
} % end fbox

\vline


\noindent\textbf{Problem 1}
Using the Baby DES cryptosystem with 3 rounds, encrypt the plaintexts 110011000000, and 010101000000. Use the key K = 001100011. Show the outcome of each round. More precisely, for each of the two encryptions show: 
\newline\indent\indent\indent\indent $L_0$  $R_0$
\newline\indent\indent\indent\indent $L_1$  $R_1$ , $K_1$
\newline\indent\indent\indent\indent $L_2$  $R_2$ , $K_2$
\newline\indent\indent\indent\indent and
\newline\indent\indent\indent\indent $L_3$  $R_3$ , $K_3$

\bigskip Separately (that is after presenting the above results), also show the work so that in case you did mistakes you can get partial credit.
\bigskip

	Solution: 
\indent\begin{tabular}{|c|c|c|}
\hline
$ $ & $plaintext 1$ & $plaintext 2$ \\
\hline
$ $  & $110011000000$ & $010101000000$\\
$ $ & $ $ & $ $ \\
$K_1$ & $00110001$ & $00110001$ \\
$K_2$ & $01100011$ & $01100011$ \\
$K_3$ & $11000110$ & $11000110$ \\
$ $ & $ $ & $ $ \\
$L_0$ & $110011$ & $010101$ \\
$R_0$ & $000000$ & $000000$ \\
$ $ & $ $ & $ $ \\
$L_1$ & $000000$ & $000000$ \\
$R_1$ & $000011$ & $100101$ \\
$ $ & $ $ & $ $ \\
$L_2$ & $000011$ & $100101$ \\
$R_2$ & $111100$ & $000000$ \\
$ $ & $ $ & $ $ \\
$L_3$ & $111100$ & $000000$ \\
$R_3$ & $110011$ & $100110$ \\
\hline
\end{tabular}

\bigskip
So after 3 rounds, the plaintext 1100 1100 0000 becomes the encrypted message, $L_3R_3$ = 1111 0011 0011 \\
So after 3 rounds, the plaintext 0101 0100 0000 becomes the encrypted message, $L_3R_3$ = 0000 0010 0110 \\

\bigskip In order to get the above results, we need to note a couple of parameters.  \\
XOR chart:  
\indent\begin{tabular}{|c|c c|}
\hline
$ \oplus $ & $ 0 $ & $ 1 $ \\
\hline
$ 0 $ &  $ 0 $ & $ 1 $ \\
$ 1 $ & $ 1 $ & $ 0 $ \\
\hline
\end{tabular} \\

and the S-boxes: \\

\begin{tabular}{|c| c c c c c c c c|}
\hline
$ S_1 $ & $ 0 $ & $ 1 $ & $ 2 $ & $ 3 $ & $ 4 $ & $ 5 $ & $ 6 $ & $ 7 $ \\
\hline
$ 0 $ & $ 101 $ & $ 010 $ & $ 001 $ & $ 110 $ & $ 011 $ & $ 100 $ & $ 111 $ & $ 000 $ \\
$ 1 $ & $ 001 $ & $ 100 $ & $ 110 $ & $ 010 $ & $ 000 $ & $ 111 $ & $ 101 $ & $ 011 $ \\
\hline 
\end{tabular} \bigskip

\begin{tabular}{|c| c c c c c c c c|}
\hline
$ S_2 $ & $ 0 $ & $ 1 $ & $ 2 $ & $ 3 $ & $ 4 $ & $ 5 $ & $ 6 $ & $ 7 $ \\
\hline
$ 0 $ & $ 100 $ & $ 000 $ & $ 110 $ & $ 101 $ & $ 111 $ & $ 001 $ & $ 011 $ & $ 010 $ \\
$ 1 $ & $ 101 $ & $ 011 $ & $ 000 $ & $ 111 $ & $ 110 $ & $ 010 $ & $ 001 $ & $ 100 $ \\
\hline
\end{tabular} \\



\bigskip To start, we need to establish the Keys such that we get $K_1$, $K_2$, and $K_3$. \\
\indent The 9 bits of the key are: 001 100 011 \\
\indent To get $K_1$ we use the 8 bits to the left so we get: 0011 0001 \\
\indent To get $K_2$ we rotate the key by 1 bit and use the 8 bits to the left, so we get: 0110 0011 \\
\indent To get $K_3$ we rotate the key again by 1 bit and use the 8 bits to the left, so we get: 1100 0110 \\


Let's start with the first plaintext. \\ 

key: 001100011 \\
message: 110011000000 \\

\textbf{Round 1 ($i = 0$) } \\

$L_0$ = 110 011 ,
$R_0$ = 000 000 , and 
$K_1$ = 0011 0001 \\
$E(R_0)$ = 0000 0000 \\
$E(R_0)  \oplus K_1 $  = 0000 0000 $ \oplus $ 0011 0001 = 0011 0001 \\
$S_1$(0011) = 110 \\
$S_2$(0001) = 000 \\
$f(R_0,K_1)$ = 110 000 \\
$f(R_0,K_1)  \oplus  L_0$ = 110 000 $ \oplus $ 110 011 = 000 011 \\
$L_i = R_{i-1} = L_1 = R_0$ = 000 000\\
$R_i = L_{i-1} \oplus f(R_{i-1},K_i) = L_0 \oplus f(R_0,K_1) = R_1$ = 000 011 \\


\textbf{Round 2 ($i = 1$) } \\

$L_1$ = 000 000 ,
$R_1$ = 000 011 , and 
$K_2$ = 0110 0011 \\
$E(R_1)$ = 0000 0011 \\
$E(R_1)  \oplus K_2 $  = 0000 0011 $ \oplus $ 0110 0011 = 0110 0000 \\
$S_1$(0110) = 111 \\
$S_2$(0000) = 100 \\
$f(R_1,K_2)$ = 111 100 \\
$f(R_1,K_2)  \oplus  L_1$ = 111 100 $ \oplus $ 000 000 = 111 100 \\
$L_i = R_{i-1} = L_2 = R_1$ = 000 011\\
$R_i = L_{i-1} \oplus f(R_{i-1},K_i) = L_1 \oplus f(R_1,K_2) = R_2$ = 111 100 \\


\textbf{Round 3 ($i = 2$) } \\

$L_2$ = 000 011 ,
$R_2$ = 111 100 , and 
$K_3$ =  1100 0110 \\
$E(R_2)$ = 1111 1100 \\
$E(R_2)  \oplus K_3 $  = 1111 1100 $ \oplus $ 1100 0110 = 0011 1010 \\
$S_1$(0011) = 110 \\
$S_2$(1010) = 000 \\
$f(R_2,K_3)$ = 110 000 \\
$f(R_2,K_3)  \oplus  L_2$ = 110 000 $ \oplus $ 000 011 = 110 011 \\
$L_i = R_{i-1} = L_3 = R_2$ = 111 100 \\
$R_i = L_{i-1} \oplus f(R_{i-1},K_i) = L_2 \oplus f(R_2,K_3) = R_3$ = 110 011 \\

So after 3 rounds, the plaintext 1100 1100 0000 becomes the encrypted message, $L_3R_3$ = 1111 0011 0011 \\

Now let's do the second plaintext. \\ 

key: 001100011 \\
message: 010101000000 \\

\textbf{Round 1 ($i = 0$) } \\

$L_0$ = 010 101 ,
$R_0$ = 000 000 , and 
$K_1$ = 0011 0001 \\
$E(R_0)$ = 0000 0000 \\
$E(R_0)  \oplus K_1 $  = 0000 0000 $ \oplus $ 0011 0001 = 0011 0001 \\
$S_1$(0011) = 110 \\
$S_2$(0001) = 000 \\
$f(R_0,K_1)$ = 110 000 \\
$f(R_0,K_1)  \oplus  L_0$ = 110 000 $ \oplus $ 010 101 = 100 101 \\
$L_i = R_{i-1} = L_1 = R_0$ = 000 000\\
$R_i = L_{i-1} \oplus f(R_{i-1},K_i) = L_0 \oplus f(R_0,K_1) = R_1$ = 100 101 \\


\textbf{Round 2 ($i = 1$) } \\

$L_1$ = 000 000 ,
$R_1$ = 100 101 , and 
$K_2$ = 0110 0011 \\
$E(R_1)$ = 1010 1001 \\
$E(R_1)  \oplus K_2 $  = 1010 1001 $ \oplus $ 0110 0011 = 1100 1010 \\
$S_1$(1100) = 000 \\
$S_2$(1010) = 000 \\
$f(R_1,K_2)$ = 000 000 \\
$f(R_1,K_2)  \oplus  L_1$ = 000 000 $ \oplus $ 000 000 = 000 000 \\
$L_i = R_{i-1} = L_2 = R_1$ = 100 101\\
$R_i = L_{i-1} \oplus f(R_{i-1},K_i) = L_1 \oplus f(R_1,K_2) = R_2$ = 000 000 \\


\textbf{Round 3 ($i = 2$) } \\

$L_2$ = 100 101 ,
$R_2$ = 000 000 , and 
$K_3$ =  1100 0110 \\
$E(R_2)$ = 0000 0000 \\
$E(R_2)  \oplus K_3 $  = 0000 0000 $ \oplus $ 1100 0110 = 1100 0110 \\
$S_1$(1100) = 000 \\
$S_2$(0110) = 011 \\
$f(R_2,K_3)$ = 000 011 \\
$f(R_2,K_3)  \oplus  L_2$ = 000 011 $ \oplus $ 100 101 = 100 110 \\
$L_i = R_{i-1} = L_3 = R_2$ = 000 000 \\
$R_i = L_{i-1} \oplus f(R_{i-1},K_i) = L_2 \oplus f(R_2,K_3) = R_3$ = 100 110 \\

So after 3 rounds, the plaintext 0101 0100 0000 becomes the encrypted message, $L_3R_3$ = 0000 0010 0110 \\
 \\


\bigskip\noindent\textbf{Problem 2} In this part you are asked to execute the differential attack on Baby DES with 3 rounds that we discussed in class. For that use the above two plaintexts  (i.e., 110011000000 and 010101000000) and the ciphertexts that you have obtained in part (1).  Next run the differential attack using the method discussed in class.  You should, of course, pretend that you do not know the key K.  Show all the steps.  Do all these for the left half and the right half and derive the possibilities for $K_3$. 
\bigskip

You need to present the following:

\begin{itemize}
	\item The 6-bit string denoted A in the description of the differential attack below.
	\item The 8-bit string E($L_3$) + E($L_3$*). 
	\item The 16 pairs of inputs having the desired XOR (for the left half and the right half), 
	\item The pairs that yield after the S-box substitutions the desired output (for the left half and the right half),
	\item List all the possibilities for $K_3$.
	\item List all the possibilities for K. If you worked correctly, the actual K should be on this list.
\end{itemize}

Organize your write-up neatly so that the grader can follow what you did.


\bigskip Solution: 
\newline $E(L3) = 1111 1100$
\newline $E(L3*) = 0000 0000$
\newline $A = 110011 + 100110 + 110011 + 010101$
\newline $= 110011$
\newline XOR of inputs to S1: 1111, XOR of outputs to s1: 110
\newline Pairs with XOR 1111:
\newline (0000,1111), (0001,1110),(0010,1101),(0011,1100),
\newline (0100,1011),(0101,1010),(0110,1001),(0111,1000),
\newline (1000,0111),(1001,0110),(1010,0101),(1011,0100),
\newline (1100,0011),(1101,0010),(1110,001),(1111,0000)
\newline DNF, no correct pairs were found


\end{document}
