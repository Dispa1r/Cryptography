\documentclass[12pt,letterpaper,final]{report}
\usepackage[utf8]{inputenc}
\usepackage{amsmath}
\usepackage{amsfonts}
\usepackage{amssymb}
\usepackage{amsthm}
\usepackage{enumitem}
\usepackage{seqsplit}
\usepackage{mathtools}
\usepackage{hyperref}
\renewcommand\qedsymbol{$\blacksquare$}
\usepackage{enumerate}


\begin{document}
\fbox{
\vbox{
\begin{flushleft}
Tanner Krebs, Gerardo Lopez, William Thornton \\  % authors' names
Math 314 \\  %class
Due: 12/5/17 \\  % date
\end{flushleft}
\center{\Large{\textbf{COSC/MATH - 314}}}
\center{\Large{\textbf{Assignment 10}}}
}  % end vbox
} % end fbox

\vline


\bigskip
\noindent\textbf{Problem 1}: Let \textit{p} be a prime that has 1024 bits and let a be a primitive root of p.
\newline Let $h(x) = a^{x}$ (mod p). We analyze if $h$ is a good hash function.
\newlist{alphalist}{enumerate}{1}
\setlist[alphalist,1]{label=\textbf{(\alph*.)}}
\begin{alphalist}
	\item\indent Is $h(x)$ preimage resistant? Say YES or NO and justify your claim.
	\newline
	\newline {\bf Solution}:
	\item\indent Is $h(x)$ weakly collision resistant? Say YES or NO and justify your claim.
	\newline
	\newline {\bf Solution}: \textit{h} is {\bf not} strongly collision-free because we can easily find $x_{i}, x_{j} \ni h(x_{i}) = h(x_{j})$ if we know a message $x_{j}$. This can be accomplished by using Fermat's Little Theorem. Since $p   \chi\alpha, \alpha^{p-q} \equiv 1$ (mod p). So,
\begin{center} 
$h(x_{i}) \equiv \alpha^{{x}_{i} }\equiv \alpha^{{x}_{i}} \cdot \alpha^{p-1} \equiv \alpha^{{x}_{i}+{p-1}}$ (mod p)
\end{center}
If we let $x_{j} = x_{i} + p -1$, we get $h(x_{i}) = h(x_{j})$ for $x_{i} \neq x_{j}$.
\end{alphalist}

\bigskip

\noindent\textbf{Problem 2}: In a family of five, what is the probability that no two people are born in the same month? Explain how you have computed the probability. 
\newline
\newline {\bf Solution:} $P(E) = (1 - \frac{1}{12})(1- \frac{2}{12})(1 -\frac{3}{12})(1 - \frac{4}{12}) \approx 38.2\%$. Furthermore, there are 12 options for the month of birth for the first person. We want the next person to have a different month of birth so there are 11 possibilities for that person. 10 possibilities for the third person. 9 for the fourth. 8 for the fifth. 
\begin{center}
So, $(12 \cdot 11 \cdot 10 \cdot 9 \cdot 8) = 95040$. 
$12^{5} = 248832$, ways to choose five peoples months of birth.
$\frac{95040}{248832} = 55/144 = 38.2\%$.
\end{center}

\bigskip

\noindent\textbf{Problem 3}: Bob is using the El Gamal signature scheme. His public key is $(p, \alpha, \beta)$ = $(97, 23, 15)$ and his secret key is $a = 67$.
\newlist{alphlist}{enumerate}{1}
\setlist[alphlist,1]{label=\textbf{(\alph*.)}}
\begin{alphlist}
	\item\indent Calculate Bob's signature for message $m = 17$ with ephemeral random $k = 31$.
	\item\indent You receive allegedly from Bob the signed message $(m_{1}, r_{1}, s_{1})$ = $(22, 37, 33)$ and $m_{2}, r_{2}, s_{2}$ = $(82, 13, 65)$. Verify is these messages originate from Bob.
\end{alphlist}


\end{document}
