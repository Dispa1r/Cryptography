\documentclass[12pt,letterpaper,final]{report}
\usepackage[utf8]{inputenc}
\usepackage{amsmath}
\usepackage{amsfonts}
\usepackage{amssymb}
\usepackage{amsthm}
\usepackage{enumitem}
\usepackage{seqsplit}
\usepackage{mathtools}
\usepackage{hyperref}
\renewcommand\qedsymbol{$\blacksquare$}
\usepackage{enumerate}


\begin{document}
\fbox{
\vbox{
\begin{flushleft}
Tanner Krebs, Gerardo Lopez, William Thornton \\  % authors' names
Math 314 \\  %class
9/14/17 \\  % date
\end{flushleft}
\center{\Large{\textbf{COSC/MATH - 314}}}
\center{\Large{\textbf{Assignment 2}}}
}  % end vbox
} % end fbox

\vline


\noindent\textbf{Problem 1: Exercise 10, page 56}
 (to convert to numbers, use a=0, b=1). Suppose there is a language that has only the letters a and b. The frequency of the letter \textit a is .1 and the frequency of \textit b is .9. A message
is encrypted using a Vigenere cipher (working mod 2 instead of mod
26). The ciphertext is BABABAAABA.

\bigskip
\newlist{alphalist}{enumerate}{1}
\setlist[alphalist,1]{label=\textbf{(\alph*.)}}
\begin{alphalist}
	\item Show that the key length is probably 2.
	\item Using the information on the frequencies of the letters, determine the key and decrypt the message.
\end{alphalist}
\bigskip  Solution: We will start by displaying different levels of displacement in hopes of finding a high number of coincidences. \\
Displacement of 1:  \indent\indent \textit{BABABAAABA} \\
\phantom{BBBBBBBBBB}  \indent\indent \textit{BABABAAABA} \\
Number of coincidences: 2 \\\\
Displacement of 2:  \indent\indent \textit{BABABAAABA} \\
\phantom{BBBBBBBBB}  \indent\indent \textit{BABABAAABA} \\
Number of coincidences: 7 \\ This is most likely going to be the key length: (Key = 2) due to the high number of frequencies.\\\\
Displacement of 3: \indent\indent \textit{BABABAAABA} \\
\phantom{BBBBBBBB} \indent\indent \textit{BABABAAABA} \\
Number of coincidences: 2 \\\\
So, 7 coincidences will be the highest amount we find. Meaning, we will further analyze every second letter in the cipher-text, starting at the first letter. With the highest percentage of B(.9) occurring most frequently (4 times) compared to A(.1) occurring \textbf{ONCE} we will assume that the first number in our key is 0. Next, analyze every second letter, starting at the second letter. \textit A occurs 5 times and \textit B zero times. We can conclude the second number in the key is 1. So the key is $\left\{0,1 \right\}$.


\bigskip
\noindent\textbf{Problem 2: Exercise 13, page 56} (if you want to compute the inverse of the matrix, see section 3.8). The ciphertext YIFZM A was encrypted by a Hill cipher with the matrix \[\begin{bmatrix} 9 &13 \\ 2 & 3   \\ \end{bmatrix} \] \\ Find the plaintext. 

\bigskip Solution: To start, let us find the inverse $[A]^{-1}$  
\bigskip So we get: 

\[[A]^{-1} =  \begin{bmatrix} 9 &13 \\ 2 & 3  \end{bmatrix}^{-1} = \frac{1}{9(3) - 13(2) }\begin{bmatrix} 3 &-13 \\ -2 & 9  \end{bmatrix} =  \frac{1}{1 }\begin{bmatrix} 3 &-13 \\ -2 & 9  \end{bmatrix}=\begin{bmatrix} 3 &-13 \\ -2 & 9  \end{bmatrix}\]
\bigskip If we split the given ciphertext by 2 letters to get smaller matrices we get: \[\begin{bmatrix} Y &I\end{bmatrix}\begin{bmatrix} F &Z\end{bmatrix}\begin{bmatrix} M &A\end{bmatrix}\] which correlates to \[\begin{bmatrix} 24 &8\end{bmatrix}\begin{bmatrix} 5 &25\end{bmatrix}\begin{bmatrix} 12 &0\end{bmatrix}\]

\bigskip Now we can start getting the plaintext: 
\[\begin{bmatrix} Y &I\end{bmatrix}\begin{bmatrix} M^{-1}\end{bmatrix} = \begin{bmatrix} 24 &8\end{bmatrix}\begin{bmatrix} 3 &-13 \\ -2 & 9  \end{bmatrix}= \begin{bmatrix} 56 &-240\end{bmatrix} mod (26) = \begin{bmatrix} 4 &20\end{bmatrix}\]
\[\begin{bmatrix} F &Z\end{bmatrix}\begin{bmatrix} M^{-1}\end{bmatrix}= \begin{bmatrix} 5 &25\end{bmatrix}\begin{bmatrix} 3 &-13 \\ -2 & 9  \end{bmatrix}= \begin{bmatrix} -35 &160\end{bmatrix} mod (26) = \begin{bmatrix} 17 &4\end{bmatrix}\]
\[\begin{bmatrix} M &A\end{bmatrix}\begin{bmatrix} M^{-1}\end{bmatrix}= \begin{bmatrix} 12 &0\end{bmatrix}\begin{bmatrix} 3 &-13 \\ -2 & 9  \end{bmatrix}= \begin{bmatrix} 36 &-156\end{bmatrix} mod (26) = \begin{bmatrix} 10 &0\end{bmatrix}\]

\bigskip Lastly if we put the results together we get: \[\begin{bmatrix} 4 &20 &17 &4 &10 &0\end{bmatrix}\] \bigskip which results in the plaintext EUREKA. 

\bigskip
\noindent\textbf{Problem 3: Exercise 14, page 56} . (Note: The matrix M has 4 entries, so there are 4 unknowns, and to determine them you need 4 equations.) 
Since the given cipher-text/plaintext pair has 6 letters, you can form 6 equations. You need to choose 4 of them, so that the system that results can be solved.) The ciphertext text GEZXDS was encrypted by a Hill cipher with a 2 x 2 matrix. The plaintext is solved. Find the encryption matrix M .

\bigskip Solution: If we split the ciphertext we get \[\begin{bmatrix} G &E\end{bmatrix}\begin{bmatrix} Z &X\end{bmatrix}\begin{bmatrix} D &S\end{bmatrix} \] which translates to \[\begin{bmatrix} 6 &4\end{bmatrix}\begin{bmatrix} 25 &23\end{bmatrix}\begin{bmatrix} 3 &18\end{bmatrix} \]

\bigskip If we split the plaintext we get \[\begin{bmatrix} S &O\end{bmatrix}\begin{bmatrix} L &V\end{bmatrix}\begin{bmatrix} E &D\end{bmatrix} \] which translates to \[\begin{bmatrix} 18 &14\end{bmatrix}\begin{bmatrix} 11 &21\end{bmatrix}\begin{bmatrix} 4 &3\end{bmatrix} \] 

\bigskip As a result, we can have 3 different matrices for [A]: \[[A] =  \begin{bmatrix} 18 &14 \\ 11 &21\end{bmatrix} or \begin{bmatrix} 18 &14 \\ 4 &3\end{bmatrix} or \begin{bmatrix} 11 &21 \\ 4 &3\end{bmatrix} \]
\bigskip In order to figure out what matrix to use, we need to find the determinant that will give us a result of 1 when we take the GCD between the det(A) and 26. In the first 2 matrices, the GCD between the determinant and 26 was 2 for both, which is incorrect. As a result we tested the last one and the GCD was indeed 1, so we found the correct matrix. 

\bigskip \[[A]^{-1} =  \begin{bmatrix} 11 &21 \\ 4 & 3  \end{bmatrix}^{-1} = \frac{1}{1 }\begin{bmatrix} 3 &-21 \\ -4 &11  \end{bmatrix} = \begin{bmatrix} 3 &-21 \\ -4 &11  \end{bmatrix} mod (26) = \begin{bmatrix} 3 &5 \\ 22 &11  \end{bmatrix}\]

\bigskip From here we use the general equation: $[A][M] = [B]$ and rearrange it so that we get $[M] = [A]^{-1}[B]$
\bigskip It is important to note that $[B]$ is found based on the letters from the ciphertext corresponding to the plaintext we chose for $[A]$. Meaning that  \[[B] =  \begin{bmatrix} 25 &23 \\ 3 & 18  \end{bmatrix}\] since \[\begin{bmatrix}L & V & E &D \end{bmatrix} -> \begin{bmatrix}Z & X &D &S \end{bmatrix}\]

\bigskip Now that we have all the missing pieces we can solve for [M]:
\[[M] = [A]^{-1}[B] =  \begin{bmatrix} 3 &5 \\ 22 & 11  \end{bmatrix}\begin{bmatrix} 25 &23 \\ 3 & 18 \end{bmatrix} = \begin{bmatrix} 90&159 \\ 583 &704  \end{bmatrix} mod (26) = \begin{bmatrix} 12 &3 \\ 11 &2  \end{bmatrix}\]



\bigskip
\noindent\textbf{Problem 4:} The following ciphertext has been obtained by Vigenere encryption.\\\\
\seqsplit{ocwyikoooniwugpmxwktzdwgtssayjzwyemdlbnqaaavsuwdvbrflauplooubfgqhgcscmgzlatoedcsdeidpbhtmuovpiekifpimfnoamvlpqfxejsmxmpgkccaykwfzpyuavtelwhrhmwkbbvgtguvtefjlodfefkvpxsgrsorvgtajbsauhzrzalkwuowhgedefnswmrciwcpaaavogpdnfpktdbalsisurlnpsjyeatcuceesohhdarkhwotikbroqrdfmzghgucebvgwcdqxgpbgqwlpbdaylooqdmuhbdqgmyweuik}

\bigskip
\newlist{alphlist}{enumerate}{2}
\setlist[alphalist,2]{label=\textbf{(\alph*.)}}
\begin{alphalist}
	\item Use displacement of 5 and 6. Which displacement produces the largest number of coincidences?
	\item Find the key.
	\item Find the plaintext.
\end{alphalist}

\bigskip
\indent Solution: 
\bigskip
Using  my program, I found that a displacement of 6 had a greater number of coincidences. so it was assumed to be the key length. To find the key I then used the keylength to split the ciphertext into 6 sections and did a frequency attack on each one.
 The most frequent letters for each section were found to be u, b, d, q, i, and g for each respective section. Assuming each one to correspond to e in the plaintext, the key would be 10, 3, 1, 14, 22, 24. This gives the plaintext:
\bigskip
\newline qxzmecqjrbeowbsatomocrsyvnvoubbrbsivnwqewscqvisvxwuthswkockmdaje 
\newline dyenfacrnvwcavengsevrwkhimqqswackaswixpjdardrlilabuhaalymxfoucyacdumc
\newline qwshojmkascdwyupywqwsbbnjgtaxmqsloytnrfryvvmposwccfvsnfziko
\newline jbhraxpnzanukrfdwscqrulvpasypvdvogekwmoblklthopuwxhsogjcgoncjrrhecdmr
\newline envhhcudywxhpryyxgetyrwjesdrwgoudqjtrimjwgecearhiec




\end{document}
