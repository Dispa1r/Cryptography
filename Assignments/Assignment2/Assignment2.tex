\documentclass[12pt,letterpaper,final]{report}
\usepackage[utf8]{inputenc}
\usepackage{amsmath}
\usepackage{amsfonts}
\usepackage{amssymb}
\usepackage{amsthm}
\usepackage{enumitem}
\usepackage{seqsplit}
\usepackage{mathtools}
\usepackage{hyperref}
\renewcommand\qedsymbol{$\blacksquare$}
\usepackage{enumerate}


\begin{document}
\fbox{
\vbox{
\begin{flushleft}
Tanner Krebs, Gerardo Lopez, William Thornton \\  % authors' names
Math 314 \\  %class
9/14/17 \\  % date
\end{flushleft}
\center{\Large{\textbf{COSC/MATH - 314}}}
\center{\Large{\textbf{Assignment 2}}}
}  % end vbox
} % end fbox

\vline


\noindent\textbf{Problem 1: Exercise 10, page 56}
 (to convert to numbers, use a=0, b=1). Suppose there is a language that has only the letters a and b. The frequency of the letter a is .1 and the frequency of 6 is .9. A message
is encrypted using a Vigenere cipher (working mod 2 instead of mod
26). The ciphertext is BABABAAABA.

\bigskip
\newlist{alphalist}{enumerate}{1}
\setlist[alphalist,1]{label=\textbf{(\alph*.)}}
\begin{alphalist}
	\item Show that the key length is probably 2.
	\item Using the information on the frequencies of the letters, determine the key and decrypt the message.
\end{alphalist}



\bigskip  Solution: 

\bigskip
\noindent\textbf{Problem 2: Exercise 13, page 56} (if you want to compute the inverse of the matrix, see section 3.8). The ciphertext YIFZM A was encrypted by a Hill cipher with the matrix \[\begin{bmatrix} 9 &13 \\ 2 & 3   \\ \end{bmatrix} \] \\ Find the plaintext. 

\bigskip Solution: 


\bigskip
\noindent\textbf{Problem 3: Exercise 14, page 56} . (Note: The matrix M has 4 entries, so there are 4 unknowns, and to determine them you need 4 equations.) 
Since the given cipher-text/plaintext pair has 6 letters, you can form 6 equations. You need to choose 4 of them, so that the system that results can be solved.) The ciphertext text GEZXDS was encrypted by a Hill cipher with a 2 x 2 matrix. The plaintext is solved. Find the encryption matrix M .

\bigskip Solution: 


\bigskip
\noindent\textbf{Problem 4:} The following ciphertext has been obtained by Vigenere encryption.\\\\
\seqsplit{ocwyikoooniwugpmxwktzdwgtssayjzwyemdlbnqaaavsuwdvbrflauplooubfgqhgcscmgzlatoedcsdeidpbhtmuovpiekifpimfnoamvlpqfxejsmxmpgkccaykwfzpyuavtelwhrhmwkbbvgtguvtefjlodfefkvpxsgrsorvgtajbsauhzrzalkwuowhgedefnswmrciwcpaaavogpdnfpktdbalsisurlnpsjyeatcuceesohhdarkhwotikbroqrdfmzghgucebvgwcdqxgpbgqwlpbdaylooqdmuhbdqgmyweuik}

\bigskip
\newlist{alphlist}{enumerate}{2}
\setlist[alphalist,2]{label=\textbf{(\alph*.)}}
\begin{alphalist}
	\item Use displacement of 5 and 6. Which displacement produces the largest number of coincidences?
	\item Find the key.
	\item Find the plaintext.
\end{alphalist}

\bigskip
\indent Solution: 




\end{document}