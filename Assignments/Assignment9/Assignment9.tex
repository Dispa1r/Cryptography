\documentclass[12pt,letterpaper,final]{report}
\usepackage[utf8]{inputenc}
\usepackage{amsmath}
\usepackage{amsfonts}
\usepackage{amssymb}
\usepackage{amsthm}
\usepackage{enumitem}
\usepackage{seqsplit}
\usepackage{mathtools}
\usepackage{hyperref}
\renewcommand\qedsymbol{$\blacksquare$}
\usepackage{enumerate}


\begin{document}
\fbox{
\vbox{
\begin{flushleft}
Tanner Krebs, Gerardo Lopez, William Thornton \\  % authors' names
Math 314 \\  %class
Due: 11/28/17 \\  % date
\end{flushleft}
\center{\Large{\textbf{COSC/MATH - 314}}}
\center{\Large{\textbf{Assignment 9}}}
}  % end vbox
} % end fbox

\vline


\noindent\textbf{Problem 1}: If a number n is composite but in the Miller-Rabin algorithm Test (n,a) outputs "n is probably prime" (see my notes, or the textbook page 178), then a is said to be a false Miller-Rabin witness for n. Show that 2 is a false Miller-Rabin witness strong for 2047.

$n = 2047, n - 1 = 2^{k} *  m$, m is odd
\newline $2046 = 2^{1} * 1023, k = 1, m = 1023$
\newline $2^{2} = 4, 2^{4} = 16, 2^{8} = 256, 2^{16} = 32, 2^{32} = 1024, 2^{64} = 0$, 0 repeats
\newline So, $2^{1023} = 1334$ (mod 2047)
\newline $b_1 = 1334^{2} = 713$
\newline $b_2 = 713^{2} = 713$, repeats until $b_{k-1}$, therefore the result is that n is probably prime, making a = 2 a false witness


\bigskip
\noindent\textbf{Problem 2}:  Let $p = 101$ (note that 101 is a prime number). It is known that 2 is a primitive root of 101. For any number n in the range ${1,2, ... , 100}$, we denote by $L_{2}(n)$ the value $k \epsilon {(1,2, ... , 100}$ such that $2^{k} = n$ (mod 101) (i.e., $L_{n}$ is the discrete log of n mod 101).
\newlist{alphalist}{enumerate}{1}
\setlist[alphalist,1]{label=\textbf{(\alph*.)}}
\begin{alphalist}
	\item\indent What is $L_{2}(1)$? Justify your answer. (Note: The answer $k = 0$ is not valid, because k has to be in the set ${(1,2, ... , 100}$).
	\item\indent Using the fact the $L_{2}(3) = 69$, determine $L_{2}(9)$
\end{alphalist}

a) $L_{2}(1) = 2^{k} = 1$ (mod 101)
\newline 100 is prime, so $k = 100$ using fermat's theorem
\newline b) $L_{2}(3) = 69$, so $2^{69} = 3$ (mod 101)
\newline $L_{2}(9) = 2^{k} = 9$ (mod 101)
\newline $2^{k} = 2^{69} * 2^{69} = 2^{138}$, $k = 138$

\bigskip
\noindent\textbf{Problem 3}: In the El Gamal cryptosystem, Alice and Bob use $p = 17$ and $a = 3$. Bob choses his secret to be $a = 6$, so $\beta = 15$. Alice sends the ciphertext $(r,t) = (7,6)$. Determine the plaintext m.

Decrypt: $\alpha^{k} = 7, \beta^{k} = 6$
\newline $m  = 6 * (7^{6})^{-1}$ (mod 17)
\newline $9^{-1}:$
\newline $17 = 1*9 + 8$
\newline $9 = 1*8 + 1$, gcd(17, 9) = 1
\newline $x_0 = 0, x_1 = 1, x_2 = -1, x_2 = 2$, so $9^{-1} = 2$ (mod 17)
\newline $6 * 2 = 12$ (mod 17), m = 12

\bigskip
\noindent\textbf{Problem 4}: Exercise 7, page 215 in the texbook.

$3^{x} = 3^{6} / (3^{10} * 3^{10}$
\newline $3^{x} = 3^{-14}$
\newline 137 is prime, so $3^{136} = 1$ (mod 137)
\newline $3^{x} = 3^{-14} * 3^{136} = 3^{122}$ (mod 137), x = 122

\end{document}
