
\documentclass[12pt,letterpaper,final]{report}
\usepackage[utf8]{inputenc}
\usepackage{amsmath}
\usepackage{amsfonts}
\usepackage{amssymb}
\usepackage{amsthm}
\usepackage{enumitem}
\usepackage{seqsplit}
\usepackage{mathtools}
\usepackage{hyperref}
\renewcommand\qedsymbol{$\blacksquare$}
\usepackage{enumerate}


\begin{document}
\fbox{
\vbox{
\begin{flushleft}
Tanner Krebs, Gerardo Lopez, William Thornton \\  % authors' names
Math 314 \\  %class
Due: 11/28/17 \\  % date
\end{flushleft}
\center{\Large{\textbf{COSC/MATH - 314}}}
\center{\Large{\textbf{Assignment 9}}}
}  % end vbox
} % end fbox

\vline


\noindent\textbf{Problem 1}: If a number n is composite but in the Miller-Rabin algorithm Test (n,a) outputs "n is probably prime" (see my notes, or the textbook page 178), then a is said to be a false Miller-Rabin witness for n. Show that 2 is a false Miller-Rabin witness strong for 2047.

\bigskip
\noindent\textbf{Problem 2}:  Let $p = 101$ (note that 101 is a prime number). It is known that 2 is a primitive root of 101. For any number n in the range ${1,2, ... , 100}$, we denote by $L_{2}(n)$ the value $k \epsilon {(1,2, ... , 100}$ such that $2^{k} = n$ (mod 101) (i.e., $L_{n}$ is the discrete log of n mod 101).
\newlist{alphalist}{enumerate}{1}
\setlist[alphalist,1]{label=\textbf{(\alph*.)}}
\begin{alphalist}
	\item\indent What is $L_{2}(1)$? Justify your answer. (Note: The answer $k = 0$ is not valid, because k has to be in the set ${(1,2, ... , 100}$).
	\item\indent Using the fact the $L_{2}(3) = 69$, determine $L_{2}(9)$
\end{alphalist}


\bigskip
\noindent\textbf{Problem 3}: In the El Gamal cryptosystem, Alice and Bob use $p = 17$ and $a = 3$. Bob choses his secret to be $a = 6$, so $\beta = 15$. Alice sends the ciphertext $(r,t) = (7,6)$. Determine the plaintext m.


\bigskip
\noindent\textbf{Problem 4}: Exercise 7, page 215 in the texbook.

\end{document}
