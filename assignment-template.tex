\documentclass[12pt,letterpaper,final]{report}
\usepackage[utf8]{inputenc}
\usepackage{amsmath}
\usepackage{amsfonts}
\usepackage{amssymb}
\usepackage{amsthm}
\renewcommand\qedsymbol{$\blacksquare$}
\usepackage{enumerate}

%\author{Marius Zimand}

\begin{document}
\fbox{
\vbox{
\begin{flushleft}
Anna Cooper, John Smith, Nick Curly \\  % authors' names
Math 314 \\  %class
9/10/16 \\  % date
\end{flushleft}
\center{\Large{\textbf{Assignment 1}}}
}  % end vbox
} % end fbox

\vline


\noindent\textbf{Problem 1:}  Here is the solution for problem 1.

Mathematical text is written like this $a+b = c$.

This is how we can have subscripts and superscripts: $a^2 + b^2 = c^2 + d_1$.

This is how we can write math equations on a separate line:
\[
a+b = c^2 + \log n.
\]
This is how we can write multi-line math equations on a separate line:
\[
\begin{array}{ll}
a+b & = c^2 + \log n \\
&\leq 5d + \sin x \\
& = A.
\end{array}
\]
Matrices can be written like this:
\[
A = 
\begin{bmatrix}
1 & a & b \\
2 & a^2 & b^3 \\
3 & a_3 & b_5
\end{bmatrix}
\]
\smallskip

This is how to make a table:
\medskip

\begin{tabular}{|c|c|c|}
\hline
$\delta$ & $a$ & $x$ \\
\hline
$q_{1}$ & $q_{1}$ & $ b$ \\
$q_{2}$ & $q_{1}$ & $ c$ \\
$q_{3}$ & $q_{2}$ & $ b$ \\
$q_{4}$ & $q_{3}$ & $ c$ \\
$q_{5}$ & $q_{4}$ & $ b$ \\
\hline
\end{tabular}
\medskip

Greek letters are easy to write: $\alpha$, $\beta$, $\gamma$, $\theta$, $\Theta$, $\omega$, $\Omega$, and so on. 
\bigskip

To make latex produce the text exactly how we type, we can use the verbatim environment. This is useful for example to write an algorithm in pseudo-code. Below is a short example.

\begin{verbatim}
s = 0
for i going from 1 to n

   s= s+ a[i]

end-for
\end{verbatim}
\noindent\textbf{Problem 2:} Here is the solution for problem 2.....





\bigskip

\noindent\textbf{Problem 3:} Here is the solution for problem 3. ...


\bigskip

\noindent\textbf{Problem 4:}  Here is the solution for problem 4. ...

\end{document}